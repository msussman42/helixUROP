\documentclass[]{article}
\usepackage[hmargin=1in,vmargin= 1.4in]{geometry}
\usepackage{mathtools}
\usepackage{titling}
\usepackage{graphicx}
\usepackage{amsmath,amssymb,mathrsfs,wrapfig,setspace,gensymb,placeins,bm}
\usepackage[]{algorithm}
\usepackage{algorithmic}
\usepackage{verbatim}
\usepackage{subfigure}
\usepackage{caption}
\usepackage{lscape}
\usepackage{cite}
\usepackage{scrextend}
\usepackage{xcolor}
\usepackage{empheq}
\usepackage{tcolorbox}
\usepackage{amsthm}

\newcommand{\bmw}{\mbox{\boldmath $w$\unboldmath}}
\newcommand{\bma}{\mbox{\boldmath $a$\unboldmath}}
\newcommand{\bmA}{\mbox{\boldmath $A$\unboldmath}}
\newcommand{\bmb}{\mbox{\boldmath $b$\unboldmath}}
\newcommand{\bmr}{\mbox{\boldmath $r$\unboldmath}}
\newcommand{\bms}{\mbox{\boldmath $s$\unboldmath}}
\newcommand{\bmz}{\mbox{\boldmath $z$\unboldmath}}
\newcommand{\bmg}{\mbox{\boldmath $g$\unboldmath}}
\newcommand{\bmq}{\mbox{\boldmath $q$\unboldmath}}
\newcommand{\bmu}{\mbox{\boldmath $u$\unboldmath}}
\newcommand{\bmm}{\mbox{\boldmath $m$\unboldmath}}
\newcommand{\bmj}{\mbox{\boldmath $j$\unboldmath}}
\newcommand{\bmk}{\mbox{\boldmath $k$\unboldmath}}
\newcommand{\bmx}{\mbox{\boldmath $x$\unboldmath}}
\newcommand{\bmn}{\mbox{\boldmath $n$\unboldmath}}
\newcommand{\bmt}{\mbox{\boldmath $t$\unboldmath}}
\newcommand{\bmf}{\mbox{\boldmath $f$\unboldmath}}
\newcommand{\bmF}{\mbox{\boldmath $F$\unboldmath}}
\newcommand{\bmS}{\mbox{\boldmath $S$\unboldmath}}
\newcommand{\bmG}{\mbox{\boldmath $G$\unboldmath}}
\newcommand{\bmV}{\mbox{\boldmath $V$\unboldmath}}
\newcommand{\bmI}{\mbox{\boldmath $I$\unboldmath}}
\newcommand{\bmD}{\mbox{\boldmath $D$\unboldmath}}
\newcommand{\bmOmega}{\mbox{\boldmath $\Omega$\unboldmath}}
\newcommand{\bmtau}{\mbox{\boldmath $\tau$\unboldmath}}
\newcommand{\hyptan}{\mbox{tanh}}
\newcommand{\hypcos}{\mbox{cosh}}
\newcommand{\hypsin}{\mbox{sinh}}


\usepackage{titlesec}
\titleformat{\section}{\large\bfseries}{\thesection}{1em}{}
\renewcommand\thesection{\Roman{section}}

\setlength{\droptitle}{-10em}
\title{An Eulerian-Lagrangian method for simulating the interaction of a 
 Non-Newtonian fluid with complex shaped micro-swimmers}
\author{B. Vargas-Torres,
   T. Solano, Y. Liu, K. Shoele, H. Mohammadigoushki, M. Sussman}

%% hexadecimal digits must be upper case.

\begin{document}
\maketitle
\vspace*{-10mm}
\section{Method} 
The numerical method used for the Eulerian part of
our micro-swimmer simulations is a staggared
grid, dynamic adaptive mesh refinement algorithm for 
incompressible multiphase flow.  
The dynamics of the micro-swimmer geometry, represented on a 
Lagrangian grid, is coupled to the Eulerian fluid flow by way of converting
the Lagrangian representation of the micro-swimmer to a
Moment-of-Fluid Eulerian representation.  The details of the numerical method
are presented in the following journal articles: 
\cite{dyadechko2005moment,ArientiSussman2014,peihierarchical,OHTA201966}.

It is important to note that my results correspond to a staggared grid 
formulation which is known to have better stability properties than the
collocated formulation\cite{rhie1983numerical}.  
The source of the stability problems for a collocated grid algorithm
is that the pressure projection step corresponding
to the collocated formulation has a non-empty null space in the pattern of
a checkerboard velocity distribution.

The numerical algorithm
for simulating non-Newtonian multi-fluid/multi-phase flows that was
presented in \cite{OHTA201966} was for a collocated grid.  The numerical
algorithm presented in \cite{peihierarchical} was for a staggared grid,
but only Newtonian flows were simulated in \cite{peihierarchical}.
The non-Newtonian multi-fluid/multi-phase results presented in 
this report are distinct from \cite{OHTA201966}
in that the present results were computed on a staggared grid.

The potential pitfalls of the collocated formulation notwithstanding,
in figures \ref{compare_bubble} and
\ref{compare_helix} below, it is observed that the collocated results of the
past agree, qualitatively, with the present staggared grid results.  This
is encouraging, since the present results (a) reinforce that the previous 
reported collocated results were valid and did not suffer from stability 
problems, and (b) reinforce that the present staggared grid method has 
been implemented correctly, and has a mathematical guarantee of 
stability\cite{GUITTET2015215} for future simulations.

The mathematical model governing our numerical method is given as follows:
\begin{eqnarray}
	\nabla\cdot\vec{u}=0 \label{divfree}
\end{eqnarray}
\begin{eqnarray}
	\frac{\partial(\rho\vec{u})}{\partial t}+
	\nabla\cdot(\rho\vec{u}\vec{u}^{T})=
	\nabla\cdot(-p\bmI + \bmtau) \label{momentum}
\end{eqnarray}
\begin{eqnarray}
\bmtau=\bmtau_{s}+\bmtau_{p} \label{solvent_and_polymer}
\end{eqnarray}
\begin{eqnarray}
\bmtau_{s}=2\mu_{s}\bmD=\mu_{s}(\nabla\vec{u}+\nabla\vec{u}^{T}) 
  \label{solvant_stress}
\end{eqnarray}
\begin{eqnarray}
\bmtau_{p}=\frac{\mu_{p}\bmf{s}(\bmA)}{\lambda}
	\label{polymer_stress}
\end{eqnarray}
\begin{eqnarray}
\overset{\triangledown}\bmA \equiv \frac{\partial\bmA}{\partial t} + 
  \nabla\cdot(\vec{u}\bmA)-
  \bmA\cdot\nabla\vec{u}-
  (\nabla\vec{u})^{T}\cdot\bmA=-\frac{f_{R}(\bmA)}{\lambda}
  \label{upper_convective_deriv}
\end{eqnarray}
\begin{eqnarray}
\frac{\partial\phi_{m}}{\partial t}+\vec{u}\cdot\nabla\phi_{m}=0
  \hspace{10pt} m=1,\ldots,M
\end{eqnarray}
$\vec{u}$ is the fluid velocity, $\rho$ is the density which can
have large jumps at material interfaces, $p$ is the pressure, and $\bmtau$
is a combination of the solvent stress tensor $\bmtau_{s}$ 
and polymer stress tensor, $\bmtau_{p}$.  $\mu_{s}$ is the solvent viscosity
which can have large jumps at material interfaces, $\bmD$ is the
rate of deformation tensor, $bmD=\nabla\vec{u}+\nabla\vec{u}^{T})$, 
and $\bmA$ is the configuration tensor. $\phi_{m}$ is the
level set function for material $m$ ($m=1,\ldots,M$) and
satisfies,
\begin{eqnarray}
	\phi_{m}(t,\vec{x})=\left\{ \begin{array}{cc}
	+d_{m} & \mbox{if $\vec{x}$ is in material $m$ at time $t$} \\
	-d_{m} & \mbox{if $\vec{x}$ is not in material $m$ at time $t$}.\\
\end{array}\right. \label{signeddist}
\end{eqnarray}
$d_{m}$ in (\ref{signeddist}) represents the closest distance from point 
$\vec{x}$ to the material $m$ interface.  The relaxation source term 
$f_{R}(\bmA)$ found in (\ref{upper_convective_deriv}) is defined as
follows
\begin{eqnarray}
	f_{R}(\bmA)=\left\{ \begin{array}{cc}
		\frac{\bmA}{1-\mbox{tr}(\bmA)/L^{2}}-\bmI &
		\mbox{FENE-P model} \\
		\frac{\bmA-\bmI}{1-\mbox{tr}(\bmA)/L^{2}} &
		                \mbox{FENE-CR model}
	\end{array}
	\right.
\end{eqnarray}
The polymer stress versus strain term $f_{s}(\bmA)$ found in
(\ref{polymer_stress}) is defined as follows
\begin{eqnarray}
	f_{s}(\bmA)=\left\{ \begin{array}{cc}
		\frac{\bmA}{1-\mbox{tr}(\bmA)/L^{2}}-\bmI &
		\mbox{FENE-P model} \\
		\frac{\bmA-\bmI}{1-\mbox{tr}(\bmA)/L^{2}} &
		                \mbox{FENE-CR model}
	\end{array}
	\right.
\end{eqnarray}
We refer the reader to the following references for more information on
the mathematical models for non-Newtonian 
fluids\cite{BIRD1980213,PURNODE19981}.

The (novel) results presented in this report are preliminary in that we only 
present results for the one-way coupling between the 
swimmer (prescribed Lagrangian description of its motion)
and the non-Newtonian fluid (Eulerian description of its motion).  The
prescribed swimmer motion influences the fluid motion but the fluid 
motion does not in turn effect the swimmer motion.  The swimmer (see
Figure \ref{swimmer}) undergoes rigid body motion in the form of
translation along the $z$ axis, $\vec{u}_{translate}=(0,0,w)$ and 
rotation about the $z$ axis $\vec{u}_{rotate}=(-\Omega y,\Omega x,0)$:
\begin{eqnarray}
	\frac{d\vec{x}_{swimmer}}{dt}= 
	\vec{u}_{translate}+\vec{u}_{rotate}.
\end{eqnarray}
The Eulerian fluid motion is coupled to the prescribed Lagrangian
swimmer motion by way of the no-slip condition,
\begin{eqnarray}
	\vec{u}_{fluid}(t,\vec{x})=
	\vec{u}_{translate}(\vec{x})+\vec{u}_{rotate},
\end{eqnarray}
which is satisfied by all points $\vec{x}$ on the swimmer surface at time
$t$.

\section{Results} 

Two types of results are presented in this report: (A) verification results, 
and (B) validation results.

\subsection{Verification Results: Rising gas bubble in non-Newtonian fluid}

In Figure \ref{gasbubblecompare} we compare computational results
for the rise of a gas bubble in a hybrid shear thinning and
FENE-CR liquid solution\cite{OHTA201966}.  The model parameters
for this test correspond to the parameters used in \cite{OHTA201966}
(see Figure 7 from \cite{OHTA201966}).  
The viscoelastic model is the hybrid
Shear thinning Carreau and FENE-CR model.  The specific non-Newtonian
model parameters are $n=1/2$, $L=10$, and $De=5$.
We expect to compute a bubble that attains
a very unique ``cusped-cap'' shape.

In our Figure \ref{gasbubblecompare}, we show the following 3 results:
(a) collocated results from \cite{OHTA201966},
(b) staggared quarter domain 3D resuts (mirrored across the $x-z$ and $y-z$ 
planes for comparison purposes) computed using our latest algorithm,
and 
(c) staggared 3D RZ (axisymmetric $R$-$Z$) results (revolved about the 
$z$ axis for comparison purposes).


\begin{figure}[htpb]
\centering
\includegraphics[width=0.3\textwidth]{collocated.png}
\includegraphics[width=0.3\textwidth]{staggared_rz_p434.png}
\includegraphics[width=0.3\textwidth]{staggared_qtr_p472.png}
\caption{Comparison between staggared and unstaggared algorithms for a
	gas bubble rising in a non-newtonian fluid; from left to right:
	(a) collocated results 3D, (b) staggared results, 3D axi-symmetric,
	at time $t=0.434$, (c) staggared results, 3D, quarter domain simulation
	at time $t=0.472$. \label{gasbubblecompare} }
\end{figure}


\subsection{Validation Results} 


\subsubsection{Helical swimmer in non-Newtonian fluid; collocated
 results versus staggared results}

\subsubsection{Helical swimmer in non-Newtonian fluid; 
 $12.4$ degrees pitch angle versus $47.5$ degrees pitch angle.}

\section*{Conclusions} 


\newpage
\bibliographystyle{acm}
\bibliography{references}

\end{document}

