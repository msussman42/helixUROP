\documentclass[]{article}
\usepackage[hmargin=1in,vmargin= 1.4in]{geometry}
\usepackage{mathtools}
\usepackage{titling}
\usepackage{graphicx}
\usepackage{amsmath,amssymb,mathrsfs,wrapfig,setspace,gensymb,placeins,bm}
\usepackage[]{algorithm}
\usepackage{algorithmic}
\usepackage{verbatim}
\usepackage{subfigure}
\usepackage{caption}
\usepackage{lscape}
\usepackage{cite}
\usepackage{scrextend}
\usepackage{xcolor}
\usepackage{empheq}
\usepackage{tcolorbox}
\usepackage{amsthm}

\newcommand{\bmw}{\mbox{\boldmath $w$\unboldmath}}
\newcommand{\bma}{\mbox{\boldmath $a$\unboldmath}}
\newcommand{\bmb}{\mbox{\boldmath $b$\unboldmath}}
\newcommand{\bmr}{\mbox{\boldmath $r$\unboldmath}}
\newcommand{\bms}{\mbox{\boldmath $s$\unboldmath}}
\newcommand{\bmz}{\mbox{\boldmath $z$\unboldmath}}
\newcommand{\bmg}{\mbox{\boldmath $g$\unboldmath}}
\newcommand{\bmq}{\mbox{\boldmath $q$\unboldmath}}
\newcommand{\bmu}{\mbox{\boldmath $u$\unboldmath}}
\newcommand{\bmm}{\mbox{\boldmath $m$\unboldmath}}
\newcommand{\bmj}{\mbox{\boldmath $j$\unboldmath}}
\newcommand{\bmk}{\mbox{\boldmath $k$\unboldmath}}
\newcommand{\bmx}{\mbox{\boldmath $x$\unboldmath}}
\newcommand{\bmn}{\mbox{\boldmath $n$\unboldmath}}
\newcommand{\bmt}{\mbox{\boldmath $t$\unboldmath}}
\newcommand{\bmf}{\mbox{\boldmath $f$\unboldmath}}
\newcommand{\bmF}{\mbox{\boldmath $F$\unboldmath}}
\newcommand{\bmS}{\mbox{\boldmath $S$\unboldmath}}
\newcommand{\bmG}{\mbox{\boldmath $G$\unboldmath}}
\newcommand{\bmV}{\mbox{\boldmath $V$\unboldmath}}
\newcommand{\bmI}{\mbox{\boldmath $I$\unboldmath}}
\newcommand{\bmD}{\mbox{\boldmath $D$\unboldmath}}
\newcommand{\bmOmega}{\mbox{\boldmath $\Omega$\unboldmath}}
\newcommand{\bmtau}{\mbox{\boldmath $\tau$\unboldmath}}
\newcommand{\hyptan}{\mbox{tanh}}
\newcommand{\hypcos}{\mbox{cosh}}
\newcommand{\hypsin}{\mbox{sinh}}


\usepackage{titlesec}
\titleformat{\section}{\large\bfseries}{\thesection}{1em}{}
\renewcommand\thesection{\Roman{section}}

\setlength{\droptitle}{-10em}
\title{An Eulerian-Lagrangian method for simulating the interaction of a 
 Non-Newtonian fluid with complex shaped micro-swimmers}
\author{Brandon Vargas-Torres,
   Tomas Solano, Kourosh Shoele, Hadi Mohammadigoushki, Mark Sussman}

%% hexadecimal digits must be upper case.

\begin{document}
\maketitle
\vspace*{-10mm}
\section*{Method} 
The numerical method used for the Eulerian part of
our micro-swimmer simulations is a staggared
grid, dynamic adaptive mesh refinement algorithm for 
incompressible multiphase flow.  
The dynamics of the micro-swimmer geometry, represented on a 
Lagrangian grid, is coupled to the Eulerian fluid flow by way of converting
the Lagrangian representation of the micro-swimmer to a
Moment-of-Fluid Eulerian representation.  The details of the numerical method
are presented in the following journal articles: 
\cite{dyadechko2005moment,ArientiSussman2014,peihierarchical,OHTA201966}.

It is important to note that my results correspond to a staggared grid 
formulation which is known to have better stability properties than the
collocated formulation\cite{rhie1983numerical}.  
The source of the stability problems for a collocated grid algorithm
is that the pressure projection step corresponding
to the collocated formulation has a non-empty null space in the pattern of
a checkerboard velocity distribution.

The numerical algorithm
for simulating non-Newtonian multi-fluid/multi-phase flows that was
presented in \cite{OHTA201966} was for a collocated grid.  The numerical
algorithm presented in \cite{peihierarchical} was for a staggared grid,
but only Newtonian flows were simulated in \cite{peihierarchical}.
The non-Newtonian multi-fluid/multi-phase results presented in 
this report are distinct from \cite{OHTA201966}
in that the present results were computed on a staggared grid.

The potential pitfalls of the collocated formulation notwithstanding,
in figures \ref{compare_bubble} and
\ref{compare_helix} below, it is observed that the collocated results of the
past agree, qualitatively, with the present staggared grid results.  This
is encouraging, since the present results (a) reinforce that the previous 
reported collocated results were valid and did not suffer from stability 
problems, and (b) reinforce that the present staggared grid method has 
been implemented correctly, and has a mathematical guarantee of 
stability\cite{GUITTET2015215} for future simulations.

The mathematical model governing our numerical method is given as follows:
\begin{eqnarray}
	\nabla\cdot\vec{u}=0 \label{divfree}
\end{eqnarray}
\begin{eqnarray}
	\frac{\partial(\rho\vec{u})}{\partial t}+
	\nabla\cdot(\rho\vec{u}\vec{u}^{T})=
	\nabla\cdot(-p\bmI + \bmtau) \label{momentum}
\end{eqnarray}
\begin{eqnarray}
\bmtau=\bmtau_{s}+\bmtau_{p} \label{solvent_and_polymer}
\end{eqnarray}
\begin{eqnarray}
\bmtau_{s}=2\mu_{s}\bmD=\mu_{s}(\nabla\vec{u}+\nabla\vec{u}^{T}) 
  \label{solvant_stress}
\end{eqnarray}
\begin{eqnarray}
\bmtau_{p}=\frac{\mu_{p}\bmf{s}(\bmA)}{\lambda}
	\label{polymer_stress}
\end{eqnarray}
\begin{eqnarray}
\overset{\bmA}\equiv \frac{\partial\bmA}{\partial t} + 
  \nabla\cdot{\vec{u}\bmA)-
  \bmA\cdot\nabla\vec{u}-
  (\nabla\vec{u})^{T}\cdot\bmA=-\frac{f_{R}(\bmA)}{\lambda}
  \label{upper_convective_deriv}
\end{eqnarray}
\begin{eqnarray}
\frac{\partial\phi_{m}}{\partial t}+\vec{u}\cdot\nabla\phi_{m}=0
  \hspace{10pt} m=1,\ldots,M
\end{eqnarray}
$\vec{u}$ is the fluid velocity, $\rho$ is the density which can
have large jumps at material interfaces, $p$ is the pressure, and $\bmtau$
is a combination of the solvent stress tensor $\bmtau_{s}$ 
and polymer stress tensor, $\bmtau_{p}$.  $\mu_{s}$ is the solvent viscosity
which can have large jumps at material interfaces, $\bmD$ is the
rate of deformation tensor, $bmD=\nabla\vec{u}+\nabla\vec{u}^{T})$, 
and $\bmA$ is the configuration tensor. $\phi_{m}$ is the
level set function for material $m$ ($m=1,\ldots,M$) and
satisfies,
\begin{eqnarray}
\phi_{m}(t,\vec{x})=\left( \begin{array}{cc}
	+d_{m} & \mbox{if $\vec{x}$ is in material $m$ at time $t$} \\
	-d_{m} & \mbox{if $\vec{x}$ is not in material $m$ at time $t$}.\\
\end{array} \label{signeddist}
\end{eqnarray}
$d_{m}$ in (\ref{signeddist}) represents the closest distance from point 
$\vec{x}$ to the material $m$ interface.  The relaxation source term 
$f_{R}(\bmA)$ found in (\ref{upper_convective_deriv}) is defined as
follows
\begin{eqnarray}
	f_{R}(\bmA)=\left\{ \begin{array}{cc}
		\frac{\bmA}{1-\mbox{tr}(\bmA)/L^{2}}-\bmI &
		\mbox{FENE-P model} \\
		\frac{\bmA-\bmI}{1-\mbox{tr}(\bmA)/L^{2}} &
		                \mbox{FENE-CR model}
	\end{array}
	\right.
\end{eqnarray}
The polymer stress versus strain term $f_{s}(\bmA)$ found in
(\ref{polymer_stress}) is defined as follows
\begin{eqnarray}
	f_{s}(\bmA)=\left\{ \begin{array}{cc}
		\frac{\bmA}{1-\mbox{tr}(\bmA)/L^{2}}-\bmI &
		\mbox{FENE-P model} \\
		\frac{\bmA-\bmI}{1-\mbox{tr}(\bmA)/L^{2}} &
		                \mbox{FENE-CR model}
	\end{array}
	\right.
\end{eqnarray}
We refer the reader to the following references for more information on
the mathematical models for non-Newtonian 
fluids\cite{BIRD1980213,PURNODE19981}.

The (novel) results presented in this report are preliminary in that we only 
present results for the one-way coupling between the 
swimmer (prescribed Lagrangian description of its motion)
and the non-Newtonian fluid (Eulerian description of its motion).  The
prescribed swimmer motion influences the fluid motion but the fluid 
motion does not in turn effect the swimmer motion.  The swimmer (see
Figure \ref{swimmer}) undergoes rigid body motion in the form of
translation along the $z$ axis, $\vec{u}_{translate}=(0,0,w)$ and 
rotation about the $z$ axis $\vec{u}_{rotate}=(-\Omega y,\Omega x,0)$:
\begin{eqnarray}
	\frac{\vec{x}_{swimmer}}{dt}= 
	\vec{u}_{translate}+\vec{u}_{rotate}.
\end{eqnarray}
The Eulerian fluid motion is coupled to the prescribed Lagrangian
swimmer motion by way of the no-slip condition,
\begin{eqnarray}
	\vec{u}_{fluid}(t,\vec{x})=
	\vec{u}_{translate}(\vec{x})+\vec{u}_{rotate},
\end{eqnarray}
which is satisfied by all points $\vec{x}$ on the swimmer surface at time
$t$.

\section*{Results} 

Two types of results are presented in this report: (A) verification results, 
and (B) validation results.

\subsection*{Verification Results} 

\subsubsection*{Rising gas bubble in non-Newtonian fluid; quarter
 domain simulation versus full domain} 

\subsubsection*{Rising gas bubble in non-Newtonian fluid; 3D 
 axisymmetric simulation versus 3D quarter domain simulation}

\subsection*{Validation Results} 

\subsubsection*{Rising gas bubble in non-Newtonian fluid; collocated
 results versus staggared results}

\subsubsection*{Helical swimmer in non-Newtonian fluid; collocated
 results versus staggared results}

\subsubsection*{Helical swimmer in non-Newtonian fluid; 
 $12.4$ degrees pitch angle versus $47.5$ degrees pitch angle.}

\section*{Conclusions} 

\begin{figure}[h]
\centering
\includegraphics[width=0.4\textwidth]{cryogenictank_nasa.png}
\caption*{Cryogenic Tank (Figure from Muratov et al. \cite{Muratov2011})}
%\begin{tiny}\cite{WILCOX2006}\end{tiny})}
\end{figure}

\FloatBarrier
		
\section*{History/Motivation} 
Past research of evaporation and boiling in multi-phase flows:\\
Can and Prosperetti (2012)\cite{CAN2012}
\begin{itemize}
\item sharp interface model for phase change
\item use a constant pressure assumption for vapor-liquid flows, 
  the velocity and temperature fields in the vapor are 
  ignored to simplify the computation 
\item dynamical effects of vapor approximated by spatially uniform pressure
\item however, the flow in the vapor regions can enhance phase change
\end{itemize}
Gibou et al. (2007)\cite{GIBOU2007}, 
Son et al. (1999)\cite{Son1999}, 
Villegas et al. (2016)\cite{VILLEGAS2016}, and 
Palmore \& Desjardins (2019)\cite{PALMORE2019}
\begin{itemize}
\item Flow field computed in both the liquid and the vapor.
\item sharp interface model for phase change
\item evaporation model depends on mass fraction
\item however, the methods by 
  \cite{GIBOU2007,Son1999,VILLEGAS2016,PALMORE2019} are 
  designed for atomization and spray problems or nucleate boiling problems
  that have outflow conditions prescribed on the computational domain
  boundary, in contrast to the sealed tank problem.
\end{itemize}
Kassemi et al. (2018)\cite{KASSEMI2018}
\begin{itemize}
\item flow assumed incompressible in the liquid 
  regions and compressible in the gas/vapor regions.
\item it is necessary for the method to reduce to first order in the
  compressible gas/vapor regions in order to guarantee that the gas/vapor 
  density is positive.
\item Since the gas/vapor mixture
  is treated as a compressible material, the 
  PISO method is used in order to allow for large time-steps; 
  thereby avoiding the stringent acoustic time-step constraint.
\end{itemize}
	
\section*{Low Mach Number Approach} 
Conjecture:\\
		
We conjecture that the effect of acoustic waves 
on the transition to unstable
behavior in Cryogenic fuel 
tanks is negligible when the time scale associated with 
phase change front propagation in the stable
condition 
is substantially slower than the sound speed of the gas.
In such cases we do not recommend simulating the
effect of acoustic waves
since a practical simulation approach that models
acoustic waves must take relatively large time steps
(e.g. PISO approach taken by Kassemi et al\cite{KASSEMI2018}).  
The acoustic waves
will be under-resolved in this case, and the associated numerical 
method must be
no higher than first order accurate at extremum 
(Godunov's Theorem, see also \cite{Harten1987}).
In the last reporting period,
we have developed a new low Mach-number approach in which
the gas/vapor density is assumed to be 
spatially uniform in each separate gas/vapor region,
thereby insuring the positivity of the gas
density, regardless of the order of accuracy of the
method.  In the bulk gas regions, we can switch to a space-time
spectrally accurate method for discretizing the momentum 
equations\cite{Pei2019}, which might enable one to
capture the transition to chaotic behavior, all the while taking large
time steps.  \\
		
\section*{Governing Equations} 
We start with the following equations 
(also given by Kassemi et al. (2018) \cite{KASSEMI2018}):
\begin{eqnarray*}
\parD{\rho}{t} + \nabla\cdot(\rho\bm{u}) = S_C
\end{eqnarray*}
\begin{eqnarray*}
\parD{}{t}(\rho\bm{u}) + \nabla\cdot(\rho \bm{u}\bm{u}) = 
 -\nabla p +\nabla\cdot(2\mu \mathbb{D}) + \rho\bm{g} + \bm{F}_M
\end{eqnarray*}
%D: rate of deformation tensor (D = \frac{\nabla v^T + \nabla v}{2}
\begin{eqnarray*}
\parD{}{t}(\rho E) + \nabla\cdot(\bm{u}(\rho E + p)) = 
\nabla\cdot(k\nabla T)+S_E
\end{eqnarray*}
where $\mathbb{D} = \frac{1}{2}(\nabla\bm{u}^T + \nabla\bm{u})$.\\
The liquid is treated as incompressible and the vapor as a 
compressible ideal gas.
		
\section*{Low Mach Number Treatment of Evaporation in Sealed Domains} 
Inspired by the work of Duarte et al\cite{DUARTE2015} and
Xia et al\cite{XiaLuTryggvason2018}, we have developed a Low Mach 
number algorithm for simulating the flow in the gas/vapor mixture for
a sealed tank, allowing for phase change.  
\par
For the low Mach number approach, we assume the liquid is 
incompressible and that the gas/vapor region(s) have spatially 
uniform density (i.e. the vapor is compressible, 
but we ignore acoustic waves).\\\\
In the bulk regions of the liquid, we have,
\begin{eqnarray*}
\nabla \cdot \bm{u} = 0,
\end{eqnarray*}
the momentum equation is,
\begin{eqnarray*}
\displaystyle \bm{u}_t +\bm{u}\cdot\nabla\bm{u} = 
-\frac{\nabla p}{\rho(T)} +
\frac{\nabla\cdot(2\mu \mathbb{D})}{\rho(T)}+\bm{g},
\end{eqnarray*}
where we define the density in the same way as defined in the 
Boussinesq approximation 
\begin{eqnarray*}
\rho = \rho_l - \alpha_l\rho_l(T-T_{0}).
\end{eqnarray*}
The temperature equation is,
\begin{eqnarray*}
(\rho C_V T)_t + \nabla\cdot(\rho C_V\bm{u}T) = \nabla\cdot k_l\nabla T
\end{eqnarray*}

In the bulk region(s) of the gas/vapor mixture, we used to have,
\begin{eqnarray}
\nabla \cdot \bm{u} = 0.  \label{divueqn}
\end{eqnarray}
For sealed domain problems, we replace (\ref{divueqn}) in the
bulk gas/vapor mixture region(s) with,
\begin{eqnarray*}
\nabla\cdot\bm{u} = 
-\frac{\dot{V}_{\textrm{tot}} H(\phi)}{\int_{\Omega}H(\phi)},
%%\dot{m}\big(\frac{1}{\rho_v}-\frac{1}{\rho_l}\big)\delta(\phi) 
\end{eqnarray*}
where
\begin{eqnarray*}
\displaystyle \dot{V}_{\textrm{tot}} \equiv 
\int_{\Omega_{\textrm{tank}}} 
\dot{m}\big(\frac{1}{\rho_v}-\frac{1}{\rho_l}\big)\delta(\phi),
\end{eqnarray*}
and
\begin{eqnarray*}
\left\{ \begin{array}{cc}
\phi>0 & \bm{x}\in\textrm{vapor} \\
\phi<0 & \bm{x}\in\textrm{liquid}
\end{array} \right.
\end{eqnarray*}
The momentum equation in the gas/vapor mixture region(s) is,
\begin{eqnarray*}
\displaystyle \bm{u}_t +\bm{u}\cdot\nabla\bm{u} = 
-\frac{\nabla p}{\rho(T)} +
\frac{\nabla\cdot(2\mu \mathbb{D})}{\rho(T)}+\bm{g},
\end{eqnarray*}
where we define the density in the same way as defined in the 
Boussinesq approximation 
\begin{eqnarray*}
\rho = \rho_v - \alpha_v\rho_v(T-T_{0}).
\end{eqnarray*}
The temperature equation in the gas/vapor region(s) is,
\begin{eqnarray*}
(\rho C_V T)_t + \nabla\cdot(\rho C_V\bm{u}T) = \nabla\cdot k_v\nabla T.
\end{eqnarray*}

\section*{Level Set Representation} 
\begin{minipage}[b]{0.6\textwidth}
\begin{eqnarray*}
\phi_t + s|\nabla\phi| = 0 \hspace{10pt} 
\left\{ \begin{array}{cc}
\phi>0 & \bm{x}\in\textrm{vapor} \\
\phi<0 & \bm{x}\in\textrm{liquid}
\end{array} \right.
\end{eqnarray*}
The interface velocity is 
\begin{eqnarray*}
\displaystyle s = \bm{u}_\Gamma\cdot\bm{n},
\end{eqnarray*}
where 
\begin{eqnarray*}
\bm{u_\Gamma} = 
  (\bm{u}_l\cdot\bm{n})\bm{n} +\frac{\dot{m}}{\rho_l}\bm{n} = 
  (\bm{u}_v\cdot\bm{n})\bm{n} +\frac{\dot{m}}{\rho_v}\bm{n},
\end{eqnarray*}
and the normal $\bm{n}$ points into the liquid:
\begin{eqnarray*}
\bm{n} = -\frac{\nabla\phi}{|\nabla\phi|}
\end{eqnarray*}
\end{minipage}
\begin{minipage}[b]{0.25\textwidth}
\begin{figure}[H]
\centering
\includegraphics[width=\textwidth]{Kartuzova_ullage.jpg}
\caption*{figure from Kartuzova, Kassemi (2016) \cite{Kartuzova2016}}
\end{figure}
\end{minipage}
	
\section*{ Interface Conditions } 
\begin{eqnarray*}
\bm{u}_l\cdot \bm{n}-\bm{u}_l\cdot \bm{n} = 
\dot{m}\big(\frac{1}{\rho_v}-\frac{1}{\rho_l}\big)
\hspace{15pt}
\nabla\cdot\bm{u} = 
\dot{m}\big(\frac{1}{\rho_v}-\frac{1}{\rho_l}\big)\delta(\phi) 
\mbox{ (weak form) }
\end{eqnarray*}
\begin{eqnarray*}
[p]_\Gamma = \sigma\kappa + [\bm{n}\cdot 2\mu\mathbb{D}\cdot\bm{n}]_\Gamma
\end{eqnarray*}
\begin{eqnarray*}
T = T_\Gamma \mbox{ on the interface}
\end{eqnarray*}
For the sealed tank problem, because we strictly enforce that the 
density in the gas/vapor mixture region(s)
is spatially uniform with respect to each
separate gas/vapor region, and since the liquid is 
incompressible, we require,
\begin{eqnarray*}
\displaystyle \int_{\Omega_{\textrm{tank}}} \nabla\cdot \bm{u} = 0.
\end{eqnarray*}
Therefore, we have
\begin{eqnarray*}
\displaystyle \nabla\cdot\bm{u} = 
\dot{m}\big(\frac{1}{\rho_v}-\frac{1}{\rho_l}\big)\delta(\phi) - 
\frac{\dot{V}_{\textrm{tot}} H(\phi)}{\int_{\Omega}H(\phi)},
\end{eqnarray*}
where
\begin{eqnarray*}
\displaystyle \dot{V}_{\textrm{tot}} \equiv 
\int_{\Omega_{\textrm{tank}}} 
\dot{m}\big(\frac{1}{\rho_v}-\frac{1}{\rho_l}\big)\delta(\phi),
\end{eqnarray*}
and
\begin{eqnarray*}
H(\phi)=\left\{ \begin{array}{cc}
 1 & \phi>0 \mbox{ (gas/vapor mixture) } \\
 0 & \textrm{otherwise}
 \end{array} \right.
\end{eqnarray*}
Mass conservation in the gas/vapor mixture region(s) 
is now achieved by increasing the
gas/vapor mixture density (uniformly in each gas/vapor mixture region) 
rather than the gas/vapor mixture volume:
\begin{eqnarray}
\frac{d\rho_{v}}{dt}=\rho_{v}\frac{\dot{V}_{\textrm{tot}}}{\int_{\Omega}H(\phi)}
\label{rho_v_ode}
\end{eqnarray}

Remark:
\begin{description}
\item[] In order for our new low Mach Number algorithm 
 to account for any number of separate gas/vapor region(s)
 in which mass transfer occurs on their respective boundaries,
 we use our previous developed ``material partitioning algorithm''
 \cite{SUSSMAN_OHTA2009,Sussman2012,wang2012hybrid} 
 which automatically identifies
 all separate contiguous material regions defined on a block structured
 adaptive mesh refinement grid.  See Figure \ref{partition}.
\end{description}

\begin{figure}[h]
\centering
\includegraphics[width=0.4\textwidth]{figcolor.jpg}
\caption{Illustration of the coloring of two separate interfacial segments
  on an adaptive hierarchy of grids.  Each contiguous gas/vapor region
  will have an associated spatially uniform density whose dependence
  on time is governed by (\ref{rho_v_ode}).
\label{partition}}
\end{figure}


\section*{} 
\begin{mymathbox}[title=Governing Equations - Low Mach Number Approach, colframe=blue!50!black]
	
\begin{eqnarray*}
H(\phi)=\left\{ \begin{array}{cc}
  1 & \phi>0 \mbox{ gas/vapor } \\
  0 & \phi<0 \mbox{ liquid }
 \end{array} \right.
\hspace{15pt}
\displaystyle \dot{V}_{\textrm{tot}} \equiv 
\int_{\Omega_{\textrm{tank}}} 
\dot{m}\big(\frac{1}{\rho_v}-\frac{1}{\rho_l}\big)\delta(\phi),
\end{eqnarray*}
\begin{eqnarray*}
\nabla \cdot \bm{u} =
\dot{m}\bigg(\frac{1}{\rho_v}-\frac{1}{\rho_l}\bigg)\delta(\phi) - 
\frac{\dot{V}_{\textrm{tot}} H(\phi)}{\int_{\Omega}H(\phi)}
\hspace{15pt}
\frac{d\rho_{v}}{dt}=\rho_{v}\frac{\dot{V}_{\textrm{tot}}}{\int_{\Omega}H(\phi)}
\end{eqnarray*}
\begin{eqnarray*}
\bm{u}_t +\bm{u}\cdot\nabla\bm{u} = 
\frac{-\nabla p}{\rho(T)} +
\frac{\nabla\cdot(2\mu \mathbb{D})}{\rho(T)}+\bm{g}
\end{eqnarray*}
\begin{eqnarray*}
(\rho C_V T)_t + \nabla\cdot(\rho C_V\bm{u}T) = \nabla\cdot k\nabla T
\end{eqnarray*}
\begin{eqnarray*}
\rho = \rho - \alpha\rho(T-T_{0}) \hspace{15pt}
\mathbb{D} = \frac{1}{2}(\nabla\bm{u}^T + \nabla\bm{u})
\end{eqnarray*}
\end{mymathbox}
		
\begin{mymathbox}[title=Interface Conditions, colframe=blue!50!black]
			
\begin{equation}
T = T_\Gamma\textrm{ on interface}\\
\end{equation}
\begin{equation}
[p]_\Gamma = \sigma\kappa + [\bm{n}\cdot 2\mu\mathbb{D}\cdot\bm{n}]_\Gamma\\
\end{equation}
\end{mymathbox}
	
	
\section*{Evaporation Model} 

References for Villegas/Palmore Desjardins model: 
Villegas, Alis, Lepilliez, Tanguy 2016 \cite{VILLEGAS2016} and
Palmore, Desjardins 2019 \cite{PALMORE2019}. 
\begin{mymathbox}[title=Villegas/Palmore Desjardins , colframe=blue!50!black]
		
\begin{eqnarray*}
\dot{m}= \frac{k_l\nabla T_l\cdot \bm{n} - 
k_v\nabla T_v\cdot \bm{n}}{L}= \frac{\rho_g D_m \nabla Y \cdot \bm{n}}{1-Y} 
\end{eqnarray*}
\end{mymathbox}

Reference for Kassemi model: Kassemi, Kartuzova, 
Hylton 2018 \cite{KASSEMI2018}
%%\cite{Schrage1953} (Schrage 1953)
\begin{mymathbox}[title=Kassemi, colframe=blue!50!black]
\small
\begin{eqnarray*}
\dot{m} = \frac{[k\nabla T\cdot \bm{n}]_\Gamma}{L} = 
\bigg(\frac{2\sigma}{2-\sigma}\bigg)
\sqrt{\frac{M}{2\pi R_{\textrm{univ}}} }
\bigg(\frac{P_\Gamma(T_\Gamma)}
{\sqrt{T_\Gamma}} - \frac{P_{\textsc{eos}}(T_v)}{\sqrt{T_v}}\bigg)
\end{eqnarray*}
\begin{eqnarray*}
P_\Gamma=
P_{\textrm{ref}}e^{\big[\frac{LM}{R_{\textrm{univ}}}
\big(\frac{1}{T_{\textrm{ref}}} - \frac{1}{T_\Gamma}\big) \big]}
\end{eqnarray*}
\end{mymathbox}
	
\section*{Work to be done on our Low Mach number model} 
We next will include temperature effects on gas/vapor expansion/compression
in the gas/vapor regions.  We refer to the work by 
Xia, Lu, Tryggvason 2018 \cite{XiaLuTryggvason2018}.
\begin{mymathbox}[title=volume change due to temperature, colframe=blue!50!black]
\begin{eqnarray*}
H(\phi)=\left\{ \begin{array}{cc}
  1 & \phi>0 \mbox{ gas/vapor } \\
  0 & \phi<0 \mbox{ liquid }
 \end{array} \right.
\end{eqnarray*}
\begin{eqnarray*}
\dot{V}_T = \tilde{\rho_{v}}\big(\frac{d\tilde{v}}{dT}\big)\frac{DT}{Dt}
\hspace{10pt} 
\displaystyle \dot{V}_{T,\textrm{tot}} \equiv 
\int_{\Omega_{\textrm{tank}}} 
\dot{V}_T H(\phi)
\hspace{10pt} 
\tilde{\rho_{v}}=\tilde{\rho_{v}}(p,T) 
\hspace{10pt} 
\tilde{v}\equiv\frac{1}{\tilde{\rho_{v}}}
\end{eqnarray*}
\begin{eqnarray*}
\nabla \cdot \bm{u} =
\dot{m}\bigg(\frac{1}{\rho_v}-\frac{1}{\rho_l}\bigg)\delta(\phi) - 
\frac{\dot{V}_{\textrm{tot}} H(\phi)}{\int_{\Omega}H(\phi)}+\dot{V}_T-
\frac{\dot{V}_{T,\textrm{tot}} H(\phi)}{\int_{\Omega}H(\phi)} \hspace{10pt}
\end{eqnarray*}
\end{mymathbox}

\section*{Numerical Method} 
\newcounter{stepcount}
\begin{enumerate}
\item Advection : CISL-MOF \cite{VAHAB2021}\cite{WEYMOUTH2010}
\item Phase change \cite{VAHAB2021}
\setcounter{stepcount}{\value{enumi}}
\end{enumerate}
\begin{figure}[h]
\centering
\includegraphics[width=0.4\textwidth]{Vahab_Tprobe.jpg}
\vspace{-1em}
\caption*{\small figure from Vahab et al. 2021 \cite{VAHAB2021}\\ remark: we must interpolate from the underlying grid to find $T(x_{\textrm{probe}})$}
%\begin{tiny}\cite{WILCOX2006}\end{tiny})}
\end{figure}
\vspace{-1em}
\begin{enumerate}
\setcounter{enumi}{\value{stepcount}}
\item Diffusion
\item Low-Mach projection \cite{Pei2019}\cite{VAHAB2021}
\end{enumerate}

\section*{Numerical Method (more detailed)} 
\fontsize{10pt}{8.2}\selectfont
{\color{blueaccent}1.} Conservative Advection\\
\hspace*{20pt} {\color{blueaccent}(a)} $\frac{D\phi}{Dt}=0, \quad m = 1,...,M$\quad Level Set function\\
\hspace*{20pt} {\color{blueaccent}(b)} volume fractions \texttt{+} centroids\\
\hspace*{20pt}\hspace*{20pt}\hspace*{20pt}$\frac{DF_m}{Dt}=0$ \quad directionally-split\cite{WEYMOUTH2010}\\
\hspace*{20pt}\hspace*{20pt}\hspace*{20pt}$\displaystyle \bm{x}_{m,(ijk)} \equiv \frac{\int_{\Omega_{ijk}}\bm{x}H(\phi_m)d\Omega}{\int_{\Omega_{ijk}}H(\phi_m)d\Omega}$\\
\hspace*{20pt} {\color{blueaccent}(c)} $\frac{D\bm{u}}{Dt}=0$\\
\hspace*{20pt} {\color{blueaccent}(d)} $\frac{D\bm{T}}{Dt}=0$\\
{\color{blueaccent}2.} Phase change \cite{VAHAB2021} \\
\hspace*{20pt} {\color{blueaccent}(a)} phase change velocity for evaporation $\frac{\dot{m}}{\rho_l}\bm{n}$\\
\hspace*{20pt} {\color{blueaccent}(b)} unsplit MOF advection of interface(s) undergoing phase change\\
\hspace*{20pt} {\color{blueaccent}(c)} redistribution of mass source to the vapor side of the interface\\
\hspace*{20pt} {\color{blueaccent}(d)} $\dot{V}_{\textrm{tot}}$ determined\\
\hspace*{20pt} {\color{blueaccent}(e)} $\rho_v \leftarrow \rho_v(1+\frac{\Delta t \dot{V}_{\textrm{tot}}}{|\Omega_v|})$\\
{\color{blueaccent}3.} Diffusion (both materials)\\
\hspace*{20pt}\hspace*{20pt}$\rho^{n+1}\frac{u^*-u^{advect}}{\Delta t} = \nabla\cdot(2\mu D^*)$ \\
\hspace*{20pt}\hspace*{20pt}$\rho^{n+1}C_V\frac{ T^{n+1}-T_{advect}}{\Delta t} = \nabla\cdot(k \nabla T)$ \\
\hspace*{20pt}\hspace*{20pt}$T=T_\Gamma \textrm{ where } \{\bm{x}|\phi(\bm{x})=0\}$\\
{\color{blueaccent}4.} Pressure Projection\\
\hspace*{20pt}\hspace*{20pt}$u^{n+1} = u^* - \Delta t \frac{\nabla p}{\rho^{n+1}}-\frac{\Delta t \kappa \nabla H}{\rho^{n+1}}$\\
\hspace*{20pt}\hspace*{20pt}$\nabla\cdot u^{n+1} = \dot{m}\bigg(\frac{1}{\rho_v}-\frac{1}{\rho_l}\bigg)\delta(\phi) - \frac{\dot{V}_{\textrm{tot}} H(\phi)}{\int_{\Omega}H(\phi)}$\\
Important Remarks:
\begin{description}
\item[1.] For phase change models in which the interface 
  temperature is a constant, then the
  interpolation to get the probe temperature depends on the 
  interface temperature and the probe length is $\Delta x/2$.   
  Otherwise, if the interface temperature
  is not constant (e.g. for evaporation), then the
  probe temperature does not include the interface temperature in the
  interpolation stencil, and the probe length is $\Delta x$.
\item[2.]  
$\kappa$ is approximated using the VOF Height Function 
technique\cite{SUSSMAN2003}
\end{description}

	
\section*{1d body fitted model in spherical coordinates for
  verifying the 2D/3D AMR-MOF algorithms.} 
$\displaystyle \frac{1}{r^2}\parD{}{r}(r^2\rho_g u_r) = 0$\\
$\displaystyle \parD{}{t}(\rho_g C_{pg}T)+\frac{1}{r^2}\parD{}{r}(r^2 \rho_g C_{pg} u_r T) = \frac{1}{r^2}\parD{}{r}(k_g r^2 \parD{T}{r})$\\
$\displaystyle \parD{}{t}(\rho_g Y)+\frac{1}{r^2}\parD{}{r}(r^2 \rho_g  u_r Y) = \frac{1}{r^2}\parD{}{r}(\rho_g D_m r^2 \parD{Y}{r})$\\\\

\noindent
Evaporation Model\\
\hspace*{20pt}  Villegas/Palmore Desjardins:\\
\hspace*{20pt} $\displaystyle \frac{[k\nabla T\cdot \bm{n}]_\Gamma}{L}= \frac{\rho_g D_m \nabla Y \cdot \bm{n}}{1-Y} $\\\\

\noindent
\hspace*{20pt} Kassemi:\\
\hspace*{20pt} $\displaystyle \frac{[k\nabla T\cdot \bm{n}]_\Gamma}{L} = \bigg(\frac{2\sigma}{2-\sigma}\bigg)\sqrt{\frac{M}{2\pi R_{\textrm{univ}}} }\bigg(\frac{P_\Gamma(T_\Gamma)}{\sqrt{T_\Gamma}} - \frac{P_{\textsc{eos}}(T_v)}{\sqrt{T_v}}\bigg)$\\\\
\hspace*{20pt} $P_\Gamma=P_{\textrm{ref}}e^{\big[\frac{LM}{R_{\textrm{univ}}}\big(\frac{1}{T_{\textrm{ref}}} - \frac{1}{T_\Gamma}\big) \big]}$

\section*{Verification tests} 
\begin{figure}[h]
\centering
\includegraphics[width=0.8\textwidth]{BorodulinDrop.jpg}
\caption{figure from Borodulin et al. 2017 \cite{BORODULIN2017}
  \label{experiment} }
\end{figure}
\FloatBarrier


\section*{Verification of the 1D body fitted code - Borodulin test} 
\begin{table}[H]
\begin{minipage}[b]{0.6\textwidth}
\centering
\resizebox{\columnwidth}{!}{
\begin{tabular}{ |p{2.3cm}||p{2.4cm}|p{3cm}|p{3cm}|  }
\hline
\multicolumn{4}{|c|}{Borodulin test using Villegas / Palmore+Desjardins model, $t=499.84$} \\
\hline
grid resolution& $A_{\textsf{ratio}}$ & $|A_{\textsf{ratio},h}-A_{\textsf{ratio},2h}|$ & $\log_2(\textrm{Err}_{h}/\textrm{Err}_{2h})$\\
\hline
32  & 0.62120 & -- & --\\
64  & 0.57918 & 0.04202 & --\\
128 & 0.55439 & 0.02479 & 0.761 \\
256 & 0.54082 & 0.01357 & 0.869 \\
512 & 0.53372 & 0.00710 & 0.935 \\
1024& 0.53007 & 0.00365 & 0.960 \\
\hline
\end{tabular}
}
\newline
\vspace{3em}

\centering
\resizebox{\columnwidth}{!}{
\begin{tabular}{ |p{2.3cm}||p{2.4cm}|p{3cm}|p{3cm}|  }
\hline
\multicolumn{4}{|c|}{Borodulin test using Kassemi model, $t=499.84$} \\
\hline
grid resolution& $A_{\textsf{ratio}}$ & $|A_{\textsf{ratio},h}-A_{\textsf{ratio},2h}|$ & $\log_2(\textrm{Err}_{h}/\textrm{Err}_{2h})$\\
\hline
32  & 0.62408 & -- & --\\
64  & 0.58102 & 0.04306 & --\\
128 & 0.55546 & 0.02556 & 0.752 \\
256 & 0.54141 & 0.01405 & 0.863 \\
512 & 0.53404 & 0.00737 & 0.931 \\
1024& 0.53026 & 0.00378 & 0.963\\
\hline
\end{tabular}
}
\newline
\vspace{2em}
\newline
\tiny{$A_{\textsf{ratio}}=(R_\Gamma/R_{\Gamma,0})^2$}
\end{minipage}
\hfill
\begin{minipage}[b]{0.39\textwidth}
\includegraphics[width=\textwidth]{BorodulinTest_VillegasModel.eps}
\includegraphics[width=\textwidth]{BorodulinTest_KassemiModel.eps}
\caption*{\tiny{blue: analytic quasi-steady infinite domain solution, orange: numerical result (grid 1024 cells)}}
\end{minipage}
\end{table}



\section*{Verification of the 1D body fitted code - Villegas test} 
\begin{table}[H]
\begin{minipage}[b]{0.6\textwidth}
\centering
\resizebox{\columnwidth}{!}{
\begin{tabular}{ |p{2.3cm}||p{2.4cm}|p{3cm}|p{3cm}|  }
\hline
\multicolumn{4}{|c|}{Villegas test using Villegas / Palmore+Desjardins model, $t=0.0142$ } \\
\hline
grid resolution& $A_{\textsf{ratio}}$ & $|A_{\textsf{ratio},h}-A_{\textsf{ratio},2h}|$ & $\log_2(\textrm{Err}_{h}/\textrm{Err}_{2h})$\\
\hline
32  & 0.59281 & -- & --\\
64  & 0.56038 & 0.03243 & --\\
128 & 0.54219 & 0.01819 & 0.834 \\
256 & 0.53253 & 0.00966 & 0.913 \\
512 & 0.52755 & 0.00498 & 0.956 \\
1024& 0.52501 & 0.00254 & 0.971 \\
\hline
\end{tabular}
}
\newline
\vspace{3em}

\centering
\resizebox{\columnwidth}{!}{
\begin{tabular}{ |p{2.3cm}||p{2.4cm}|p{3cm}|p{3cm}|  }
\hline
\multicolumn{4}{|c|}{Villegas test using Kassemi model, $t=0.0142$} \\
\hline
grid resolution& $A_{\textsf{ratio}}$ & $|A_{\textsf{ratio},h}-A_{\textsf{ratio},2h}|$ & $\log_2(\textrm{Err}_{h}/\textrm{Err}_{2h})$\\
\hline
32  & 0.59530 & -- & --\\
64  & 0.56192 & 0.03338 & --\\
128 & 0.54309 & 0.01883 & 0.826\\
256 & 0.53306 & 0.01003 & 0.909\\
512 & 0.52787 & 0.00519 & 0.951\\
1024& 0.52524 & 0.00263 & 0.981\\
\hline
\end{tabular}
}
\newline
\vspace{2em}
\newline
\tiny{$A_{\textsf{ratio}}=(R_\Gamma/R_{\Gamma,0})^2$}
\end{minipage}
\hfill
\begin{minipage}[b]{0.39\textwidth}
\includegraphics[width=\textwidth]{VillegasTest_VillegasModel.eps}
\includegraphics[width=\textwidth]{VillegasTest_KassemiModel.eps}
\caption*{\tiny{blue: analytic quasi-steady infinite domain solution, orange: numerical result (grid 1024 cells)}}
\end{minipage}
\end{table}


\section*{Verification of the new 2D/3D low Mach number AMR-MOF method} 
% new figures contained in plot_data folder

\begin{figure}[h]
\centering
\includegraphics[width=0.67\textwidth]{plot_data/BorodulinTest_small.png}
\caption*{Grid refinement study corresponding to experiments reported in
 Borodulin et al\cite{BORODULIN2017} for an evaporating spherical
 drop (see 3rd row of Figure \ref{experiment}).  
 Level 4 case has 5 levels total;
 effective fine grid resolution is $512\times 512$.  The ``Kassemi 1D''
 result comes from our 1D, spherical coordinate systems, body fitted code
 which we consider as the ``ground truth.'' The ``analytical''
 result corresponds to the analytical solution for the quasi steady problem }
\end{figure}
\FloatBarrier

\begin{figure}[h]
\centering
\includegraphics[width=0.67\textwidth]{plot_data/VillegasTest_small.png}
\caption*{Verification of our new low Mach number algorithm for an
 evaporating spherical drop benchmark problem posed by
 Villegas et al (2016)\cite{VILLEGAS2016} and 
 Palmore \& Desjardins (2019)\cite{PALMORE2019}.  
 All AMR-MOF results have two levels of AMR in addition to the base
 coarse grid; the effective fine grid resolution is $128\times 128$.
 The orange line corresponds to the ``Kassemi'' model with outflow (non-sealed)
 boundary conditions. The red line 
 corresponds to the ``Kassemi'' model with sealed (no outflow) 
 boundary conditions.  The aqua line corresponds to the ``Palmore
 Desjardins'' model with outflow boundary conditions.  The
 blue line corresponds to the ``Palmore
 Desjardins'' model with outflow boundary conditions using our 1D 
 body-fitted code, and the dashed line corresponds to the quasi-steady
 infinite domain analytic solution using the ``Palmore-Desjardins''
 model. }
\end{figure}
\FloatBarrier

%% radius=0.0432 (initial radius 0.05)
%% aratio= ( 0.0432/0.05 ) ^ 2=0.746  time=273 seconds
\begin{figure}[h]
\centering
\includegraphics[width=0.60\textwidth]{plot_data/zoom_out_lev4.png}
\caption*{
  View of AMR-MOF simulation results
  for the velocity field, temperature profile, and
  interface (pink curve) at time $t=273$ seconds.   The numerical
  simulation corresponds to the experiment conducted by
  Borodulin et al\cite{BORODULIN2017} for an evaporating spherical
  drop (see 3rd row of Figure \ref{experiment}).  The grid hierarchy consists
  of a base coarse level plus 4 additional levels of AMR.  The effective
  fine grid resolution is $512\times 512$.
  }
\end{figure}
\FloatBarrier

\begin{figure}[h]
\centering
\includegraphics[width=0.60\textwidth]{plot_data/zoom_in_lev4.png}
\caption*{
  Zoomed-in View of AMR-MOF simulation results
  for the velocity field, temperature profile, and
  interface (pink curve) at time $t=273$ seconds.   The numerical
  simulation corresponds to the experiment conducted by
  Borodulin et al\cite{BORODULIN2017} for an evaporating spherical
  drop (see 3rd row of Figure \ref{experiment}).  The grid hierarchy consists
  of a base coarse level plus 4 additional levels of AMR.  The effective
  fine grid resolution is $512\times 512$.
}
\end{figure}
\FloatBarrier

\newpage
\bibliographystyle{acm}
\bibliography{references}

\end{document}

